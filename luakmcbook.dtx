\NeedsTeXFormat{LaTeX2e}

%\ProvidesClass{luakmcbook}[2020/12/06 customized bxjsbook class]

% bxjsbookをもとにする
\LoadClassWithOptions{bxjsbook}

% === ここから Pandoc のテンプレからもってきたやつ
\usepackage{lmodern}
\usepackage{amssymb,amsmath}
\usepackage{ifxetex,ifluatex}
\usepackage{fixltx2e} % provides \textsubscript

\usepackage{unicode-math}
\defaultfontfeatures{Ligatures=TeX,Scale=MatchLowercase}

% use upquote if available, for straight quotes in verbatim environments
\IfFileExists{upquote.sty}{\usepackage{upquote}}{}

\usepackage{hyperref}

% set default figure placement to htbp
\makeatletter
\def\fps@figure{htbp}
\makeatother

% === ここまで Pandoc のテンプレからもってきたやつ

% 改頁なしのinclude等を提供
\usepackage{newclude}

% 欧文フォントを変更
\usepackage{fontspec}
\setmainfont[Scale=MatchLowercase]{Linux Libertine O} % \rmfamily のフォント

% 校正規約はこれがないと組めなかった
\usepackage{booktabs}
\usepackage{longtable}

% ノド側にテキストが寄るのを直す
\setlength{\textwidth}{\fullwidth}  %本文の幅(textwidth)を全体の幅(=ヘッダ部の幅)にそろえる
\setlength{\evensidemargin}{\oddsidemargin} %偶数ページの余白と奇数ページの余白をそろえる

% 部(part)の見出しを出さないようにする
\makeatletter
\renewcommand\part{%
\secdef\@part\@spart}

\def\@part[#1]#2{%
\ifnum \c@secnumdepth >-2\relax
\refstepcounter{part}%
\addcontentsline{toc}{part}{%
\prepartname\thepart\postpartname\hspace{1\jsZw}#1}%
\else
\addcontentsline{toc}{part}{#1}%
\fi
\@endpart}
\makeatother